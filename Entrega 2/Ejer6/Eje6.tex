%-------------------------------
%	DOCUMENT SETTINGS
%-------------------------------
\documentclass[a4paper]{article}

\addtolength{\hoffset}{-1.75cm}
\addtolength{\textwidth}{4.5cm}
\addtolength{\voffset}{-0.5cm}
\addtolength{\textheight}{5cm}
\setlength{\parskip}{0pt}
\setlength{\parindent}{0in}

%----------------------------------------------------------------------------------------
%	PACKAGES AND OTHER DOCUMENT CONFIGURATIONS
%----------------------------------------------------------------------------------------

\usepackage{mathtools}
\usepackage[shortlabels]{enumitem}
\DeclarePairedDelimiter\abs{\lvert}{\rvert}%
\usepackage{cancel}
\usepackage{blindtext} % Package to generate dummy text
\usepackage{charter} % Use the Charter font
\usepackage[utf8]{inputenc} % Use UTF-8 encoding
\usepackage{microtype} % Slightly tweak font spacing for aesthetics
\usepackage[english]{babel} % Language hyphenation and typographical rules
\usepackage{amsthm, amsmath, amssymb} % Mathematical typesetting
\usepackage{float} % Improved interface for floating objects
\usepackage[final, colorlinks = true, 
linkcolor = black, 
citecolor = black]{hyperref} % For hyperlinks in the PDF
\usepackage{graphicx, multicol} % Enhanced support for graphics
\usepackage{xcolor} % Driver-independent color extensions
\usepackage{marvosym, wasysym} % More symbols
\usepackage{rotating} % Rotation tools
\usepackage{censor} % Facilities for controlling restricted text
\usepackage{listings} % Environment for non-formatted code, !uses style file!
\usepackage{pseudocode} % Environment for specifying algorithms in a natural way
% Environment for f-structures, !uses style file!
\usepackage{booktabs} % Enhances quality of tables
\usepackage{tikz-qtree} % Easy tree drawing tool
% Configuration for b-trees and b+-trees, !uses style file!
%\usepackage[backend=biber,style=numeric,
%            sorting=nyt]{biblatex} % Complete reimplementation of bibliographic facilities
%\addbibresource{ecl.bib}
\usepackage{csquotes} % Context sensitive quotation facilities
\usepackage[yyyymmdd]{datetime} % Uses YEAR-MONTH-DAY format for dates
\renewcommand{\dateseparator}{-} % Sets dateseparator to '-'
\usepackage{fancyhdr} % Headers and footers
\pagestyle{fancy} % All pages have headers and footers
\fancyhead{}\renewcommand{\headrulewidth}{0pt} % Blank out the default header
\fancyfoot[L]{} % Custom footer text
\fancyfoot[C]{} % Custom footer text
\fancyfoot[R]{\thepage} % Custom footer text
\newcommand{\note}[1]{\marginpar{\scriptsize \textcolor{red}{#1}}} % Enables comments in red on margin

%----------------------------------------------------------------------------------------
%	CUSTOM COMMANDS
%----------------------------------------------------------------------------------------

\newcommand{\R}{\mathbb{R}}
\newcommand{\E}{\mathbb{E}}
\newcommand{\E}{\mathbb{I}}
\newcommand{\Var}{\text{Var}}
% Para poner sonrisa sobre puntos suspensivos. Uso: \overplace{n}{\dotsc}
\newcommand{\overplace}[2]{%
	\overset{\substack{#1\\\smile}}{#2}%
}

\begin{document}
	
%-------------------------------
%	TITLE SECTION
%-------------------------------

\fancyhead[C]{}
\hrule \medskip % Upper rule
\begin{minipage}{0.295\textwidth} 
	\raggedright
	\footnotesize
	José Antonio Álvarez Ocete \hfill\\   
	77553417Q \hfill\\
	joseantonio.alvarezo@estudiante.uam.es
\end{minipage}
\begin{minipage}{0.4\textwidth} 
	\centering 
	\large 
	Entrega de problemas Boostrap\\ 
	\normalsize 
	Métodos Avanzados en Estadística\\ 
\end{minipage}
\begin{minipage}{0.295\textwidth} 
	\raggedleft
	\today\hfill\\
\end{minipage}
\medskip\hrule 
\bigskip

%-------------------------------
%	CONTENTS
%-------------------------------

\section*{Ejercicio 6.}

\textbf{Enunciado.} Sean $X_1, \ldots, X_n$ variables aleatorias \emph{i.i.d.} de una distribución con densidad $f$. Se considera el estimador del núcleo $çhat f$ con núcleo rectangular $K(x) = \I_{[−1/2,1/2]}(x)$ y parámetro de suavizado $h$.

\begin{enumerate}[a)]
	\item Calcula el sesgo y la varianza de $\hat f(x)$, para un valor de $x$ fijo.
	
	\item Demuestra que tanto el sesgo como la varianza tienden a cero si $h \rightarrow 0$ y $nh \rightarrow \infty$.
\end{enumerate}

Al estudiar el sesgo y la varianza de $\hat f(x)$ para un valor de $x$ fijo, estudiamos dichos valores respecto a la muestra tomada de  $X_1, \ldots, X_n$. En primer lugar, calculemos la esperanza del núcleo rectangular dado por la función indicatriz:

\[
	\E \bigg[ K(\frac{x - t}{h}) \bigg] = \int_{-\infty}^{+\infty} f(t) \; K(\frac{x - t}{h}) \; dt 
\]

Utilizamos el cambio de variable $w = \frac{x-t}{h}, dw = -\frac{dt}{h}$ e invirtiendo los límites de integración obtenemos:

\begin{align}
	\E \bigg[ K(\frac{x - t}{h}) \bigg] & = \int_{-\infty}^{+\infty} f(t) \; K(\frac{x - t}{h}) \; dt \\
	& = \int_{+\infty}^{-\infty} f(x - wh) \; K(w) \; (-h) \; dw \\
	& = h \int_{-\infty}^{+\infty} f(x - wh) \; \I_{[−1/2,1/2]}(w) \; dw \\
	& = h \int_{-\frac{1}{2}}^{\frac{1}{2}} f(x - wh) dw \\
\end{align}

Calculémos el sesgo de nuestro estimador del núcleo haciendo uso de la expresión anterior:



\end{document}
