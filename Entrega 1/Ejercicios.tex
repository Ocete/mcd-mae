\documentclass[a4paper]{article} 
\addtolength{\hoffset}{-2.25cm}
\addtolength{\textwidth}{4.5cm}
\addtolength{\voffset}{-3.25cm}
\addtolength{\textheight}{5cm}
\setlength{\parskip}{0pt}
\setlength{\parindent}{0in}

%----------------------------------------------------------------------------------------
%	PACKAGES AND OTHER DOCUMENT CONFIGURATIONS
%----------------------------------------------------------------------------------------

\usepackage{blindtext} % Package to generate dummy text
\usepackage{charter} % Use the Charter font
\usepackage[utf8]{inputenc} % Use UTF-8 encoding
\usepackage{microtype} % Slightly tweak font spacing for aesthetics
\usepackage[english, ngerman]{babel} % Language hyphenation and typographical rules
\usepackage{amsthm, amsmath, amssymb} % Mathematical typesetting
\usepackage{float} % Improved interface for floating objects
\usepackage[final, colorlinks = true, 
            linkcolor = black, 
            citecolor = black]{hyperref} % For hyperlinks in the PDF
\usepackage{graphicx, multicol} % Enhanced support for graphics
\usepackage{xcolor} % Driver-independent color extensions
\usepackage{marvosym, wasysym} % More symbols
\usepackage{rotating} % Rotation tools
\usepackage{censor} % Facilities for controlling restricted text
\usepackage{listings} % Environment for non-formatted code, !uses style file!
\usepackage{pseudocode} % Environment for specifying algorithms in a natural way
 % Environment for f-structures, !uses style file!
\usepackage{booktabs} % Enhances quality of tables
\usepackage{tikz-qtree} % Easy tree drawing tool
 % Configuration for b-trees and b+-trees, !uses style file!
%\usepackage[backend=biber,style=numeric,
%            sorting=nyt]{biblatex} % Complete reimplementation of bibliographic facilities
%\addbibresource{ecl.bib}
\usepackage{csquotes} % Context sensitive quotation facilities
\usepackage[yyyymmdd]{datetime} % Uses YEAR-MONTH-DAY format for dates
\renewcommand{\dateseparator}{-} % Sets dateseparator to '-'
\usepackage{fancyhdr} % Headers and footers
\pagestyle{fancy} % All pages have headers and footers
\fancyhead{}\renewcommand{\headrulewidth}{0pt} % Blank out the default header
\fancyfoot[L]{} % Custom footer text
\fancyfoot[C]{} % Custom footer text
\fancyfoot[R]{\thepage} % Custom footer text
\newcommand{\note}[1]{\marginpar{\scriptsize \textcolor{red}{#1}}} % Enables comments in red on margin

\newcommand{\R}{\mathbb{R}}
\newcommand{\E}{\mathbb{E}}
\newcommand{\Var}{\text{Var}}
% Para poner sonrisa sobre puntos suspensivos. Uso: \overplace{n}{\dotsc}
\newcommand{\overplace}[2]{%
	\overset{\substack{#1\\\smile}}{#2}%
}

%----------------------------------------------------------------------------------------
% Generado por Rstudio al transformar desde Rmd:

\usepackage{lmodern}
\usepackage{amssymb,amsmath}
\usepackage{ifxetex,ifluatex}
\ifnum 0\ifxetex 1\fi\ifluatex 1\fi=0 % if pdftex
\usepackage[T1]{fontenc}
\usepackage[utf8]{inputenc}
\usepackage{textcomp} % provide euro and other symbols
\else % if luatex or xetex
\usepackage{unicode-math}
\defaultfontfeatures{Scale=MatchLowercase}
\defaultfontfeatures[\rmfamily]{Ligatures=TeX,Scale=1}
\fi
% Use upquote if available, for straight quotes in verbatim environments
\IfFileExists{upquote.sty}{\usepackage{upquote}}{}
\IfFileExists{microtype.sty}{% use microtype if available
	\usepackage[]{microtype}
	\UseMicrotypeSet[protrusion]{basicmath} % disable protrusion for tt fonts
}{}
\makeatletter
\@ifundefined{KOMAClassName}{% if non-KOMA class
	\IfFileExists{parskip.sty}{%
		\usepackage{parskip}
	}{% else
		\setlength{\parindent}{0pt}
		\setlength{\parskip}{6pt plus 2pt minus 1pt}}
}{% if KOMA class
	\KOMAoptions{parskip=half}}
\makeatother
\usepackage{xcolor}
\IfFileExists{xurl.sty}{\usepackage{xurl}}{} % add URL line breaks if available
\IfFileExists{bookmark.sty}{\usepackage{bookmark}}{\usepackage{hyperref}}
\hypersetup{
	pdftitle={Ejercicio 7},
	pdfauthor={José Antonio Álvarez Ocete},
	hidelinks,
	pdfcreator={LaTeX via pandoc}}
\urlstyle{same} % disable monospaced font for URLs
\usepackage[margin=1in]{geometry}
\usepackage{color}
\usepackage{fancyvrb}
\newcommand{\VerbBar}{|}
\newcommand{\VERB}{\Verb[commandchars=\\\{\}]}
\DefineVerbatimEnvironment{Highlighting}{Verbatim}{commandchars=\\\{\}}
% Add ',fontsize=\small' for more characters per line
\usepackage{framed}
\definecolor{shadecolor}{RGB}{248,248,248}
\newenvironment{Shaded}{\begin{snugshade}}{\end{snugshade}}
\newcommand{\AlertTok}[1]{\textcolor[rgb]{0.94,0.16,0.16}{#1}}
\newcommand{\AnnotationTok}[1]{\textcolor[rgb]{0.56,0.35,0.01}{\textbf{\textit{#1}}}}
\newcommand{\AttributeTok}[1]{\textcolor[rgb]{0.77,0.63,0.00}{#1}}
\newcommand{\BaseNTok}[1]{\textcolor[rgb]{0.00,0.00,0.81}{#1}}
\newcommand{\BuiltInTok}[1]{#1}
\newcommand{\CharTok}[1]{\textcolor[rgb]{0.31,0.60,0.02}{#1}}
\newcommand{\CommentTok}[1]{\textcolor[rgb]{0.56,0.35,0.01}{\textit{#1}}}
\newcommand{\CommentVarTok}[1]{\textcolor[rgb]{0.56,0.35,0.01}{\textbf{\textit{#1}}}}
\newcommand{\ConstantTok}[1]{\textcolor[rgb]{0.00,0.00,0.00}{#1}}
\newcommand{\ControlFlowTok}[1]{\textcolor[rgb]{0.13,0.29,0.53}{\textbf{#1}}}
\newcommand{\DataTypeTok}[1]{\textcolor[rgb]{0.13,0.29,0.53}{#1}}
\newcommand{\DecValTok}[1]{\textcolor[rgb]{0.00,0.00,0.81}{#1}}
\newcommand{\DocumentationTok}[1]{\textcolor[rgb]{0.56,0.35,0.01}{\textbf{\textit{#1}}}}
\newcommand{\ErrorTok}[1]{\textcolor[rgb]{0.64,0.00,0.00}{\textbf{#1}}}
\newcommand{\ExtensionTok}[1]{#1}
\newcommand{\FloatTok}[1]{\textcolor[rgb]{0.00,0.00,0.81}{#1}}
\newcommand{\FunctionTok}[1]{\textcolor[rgb]{0.00,0.00,0.00}{#1}}
\newcommand{\ImportTok}[1]{#1}
\newcommand{\InformationTok}[1]{\textcolor[rgb]{0.56,0.35,0.01}{\textbf{\textit{#1}}}}
\newcommand{\KeywordTok}[1]{\textcolor[rgb]{0.13,0.29,0.53}{\textbf{#1}}}
\newcommand{\NormalTok}[1]{#1}
\newcommand{\OperatorTok}[1]{\textcolor[rgb]{0.81,0.36,0.00}{\textbf{#1}}}
\newcommand{\OtherTok}[1]{\textcolor[rgb]{0.56,0.35,0.01}{#1}}
\newcommand{\PreprocessorTok}[1]{\textcolor[rgb]{0.56,0.35,0.01}{\textit{#1}}}
\newcommand{\RegionMarkerTok}[1]{#1}
\newcommand{\SpecialCharTok}[1]{\textcolor[rgb]{0.00,0.00,0.00}{#1}}
\newcommand{\SpecialStringTok}[1]{\textcolor[rgb]{0.31,0.60,0.02}{#1}}
\newcommand{\StringTok}[1]{\textcolor[rgb]{0.31,0.60,0.02}{#1}}
\newcommand{\VariableTok}[1]{\textcolor[rgb]{0.00,0.00,0.00}{#1}}
\newcommand{\VerbatimStringTok}[1]{\textcolor[rgb]{0.31,0.60,0.02}{#1}}
\newcommand{\WarningTok}[1]{\textcolor[rgb]{0.56,0.35,0.01}{\textbf{\textit{#1}}}}
\usepackage{graphicx,grffile}
\makeatletter
\def\maxwidth{\ifdim\Gin@nat@width>\linewidth\linewidth\else\Gin@nat@width\fi}
\def\maxheight{\ifdim\Gin@nat@height>\textheight\textheight\else\Gin@nat@height\fi}
\makeatother
% Scale images if necessary, so that they will not overflow the page
% margins by default, and it is still possible to overwrite the defaults
% using explicit options in \includegraphics[width, height, ...]{}
\setkeys{Gin}{width=\maxwidth,height=\maxheight,keepaspectratio}
% Set default figure placement to htbp
\makeatletter
\def\fps@figure{htbp}
\makeatother
\setlength{\emergencystretch}{3em} % prevent overfull lines
\providecommand{\tightlist}{%
	\setlength{\itemsep}{0pt}\setlength{\parskip}{0pt}}
\setcounter{secnumdepth}{-\maxdimen} % remove section numbering


\usepackage{mathtools}
\usepackage{amsmath}
\usepackage[shortlabels]{enumitem}
\DeclarePairedDelimiter\abs{\lvert}{\rvert}%
\usepackage{cancel}

\begin{document}

%-------------------------------
%	TITLE SECTION
%-------------------------------

\fancyhead[C]{}
\hrule \medskip % Upper rule
\begin{minipage}{0.295\textwidth} 
\raggedright
\footnotesize
José Antonio Álvarez Ocete \hfill\\   
77553417Q \hfill\\
joseantonio.alvarezo@estudiante.uam.es
\end{minipage}
\begin{minipage}{0.4\textwidth} 
\centering 
\large 
Ejercicios\\ 
\normalsize 
Métodos Avanzados en Estadística\\ 
\end{minipage}
\begin{minipage}{0.295\textwidth} 
\raggedleft
\today\hfill\\
\end{minipage}
\medskip\hrule 
\bigskip
	
%-------------------------------
%	CONTENTS
%-------------------------------

\section*{Ejercicio 2.}

\textbf{Enunciado.} Sea $X_1, \ldots, X_n$ una muestra de $n$ observaciones \emph{iid} de una distribución $F$ con esperanza $\mu$ y varianza $\sigma^2$, y sea $X_1^*, \ldots, X_n^*$ una muestra de $n$ observaciones de la distribución empírica de la muestra original $F_n$. Calcula las siguientes cantidades: \\

\begin{enumerate}[a)]
	\item $\E_{F_n}(\bar X_n^*) := \E(\bar X_n^* | X_1, \ldots, X_n)$
	
	Esta esperanza nos supone un cálculo directo utilizando la linealidad de la esperanza:
	
	\[
		\E_{F_n}(\bar X_n^*) = \E_{F_n} \bigg( \frac{1}{n} \sum_{i=1}^n X_i^* \bigg) = \frac{1}{n} \sum_{i=1}^n \E_{F_n}(X_i^*) \stackrel{\text{iid}}{=} \frac{1}{n} \cdot n \ \E_{F_n}(X_1^*) = \bar x
	\]
	
	\item $\E_{F}(\bar X_n^*)$
	
	Por un proceso análogo al anterior podemos obtener que:
	
	\[
		\E_{F}(\bar X_n^*) = \E_{F} \bigg( \frac{1}{n} \sum_{i=1}^n X_i^* \bigg) = \frac{1}{n} \sum_{i=1}^n \E_{F}(X_i^*) \stackrel{\text{iid}}{=} \frac{1}{n} \cdot n \ \E_{F}(X_1^*) = \E_{F}(X_1^*)
	\]
	
	Sin embargo, $\E_{F}(X_1^*)$ no se puede calcular directamente pues depende de la muestra tomada $X_1, \ldots, X_n$. Podemos utilizar la \textbf{ley de la esperanza iterada}$\E[Y] = \E(\E(Y|X))$ y el valor calculado en el apartado anterior:
		
	\[
		\E_{F}(X_1^*) = \E_{F}(\E(X_1^* | X_1, \ldots, X_n)) = \E_{F}(\E_{F_n}(X_1^*)) = \E_{F}(\bar x) = \mu
	\]
	
	\item $\Var_{F_n}(\bar X_n^*) := Var(\bar X_n^* | X_1, \ldots, X_n)$
	
	Expresaremos esta varianza en función de la varianza muestra insesgada:
	
	\[
		s^2 = \frac{1}{n-1} \bigg( \sum_i x_i^2 - n \bar x^2 \bigg) = \frac{n}{n-1} \bigg( \frac{1}{n}\sum_i x_i^2 - \bar x^2 \bigg)
	\]
	
	Al ser insesgada, sabemos que $\E[s^2] = \sigma^2$. Esto nos será útil en el último apartado. Procedamos con el cálculo de la varianza:
		
	\[
		\begin{split}
			\Var_{F_n}(\bar X_n^*) & \stackrel{\text{iid}}{=} \frac{1}{n^2} \sum_{i=1}^n \Var_{F_n}(X_i^*) \\
			& \stackrel{\text{iid}}{=} \frac{1}{n^2} \cdot n \ \Var_{F_n}(X_1^*) \\
			& = \frac{1}{n} \bigg( \E_{F_n}( {X_1^*}^2 ) - \E_{F_n}( X_1^* )^2 \bigg) \\
			& = \frac{1}{n} \bigg( \frac{1}{n} \sum_{i=1}^n X_i^2 - \bar x^2 \bigg) \\
			& = \frac{n-1}{n^2} \underbrace{\frac{n}{n-1} \bigg( \frac{1}{n} \sum_{i=1}^n X_i^2 - \bar x^2 \bigg)}_{s^2} \\
			& = \frac{n-1}{n^2} s^2 \\
		\end{split}
	\]
	
	\item $Var_{F}(\bar X_n^*)$
	
	Para este último apartado haremos uso de la \textbf{ley de la varianza total}:
	
	\[
		\Var(Y) = \E(\Var(Y|X)) + \Var(\E(Y|X))
	\]
	
	Procedemos con el cálculo de la varianza:
	
	\[
		\begin{split}
			\Var_{F}(\bar X_n^*) & = \E_{F}( \underbrace{\Var_{F_n}(\bar X^*)}_{\frac{n-1}{n^2} s^2} ) + \Var_{F}( \underbrace{\E_{F_n}(\bar X^*)}_{\bar x} )\\
			& = \frac{n-1}{n^2} \E_{F}(s^2) + \Var_{F}( \bar x ) \\
			& \stackrel{\text{iid}}{=} \frac{n-1}{n^2} \sigma^2 + \frac{1}{n} \Var_{F}(X_1)  \\
			& = \frac{n-1}{n^2} \sigma^2 + \frac{1}{n} \sigma^2 \\
			& = \frac{2n-1}{n^2} \sigma^2 \\
		\end{split}
	\]
	
\end{enumerate}

\end{document}
