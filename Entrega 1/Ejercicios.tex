\documentclass[a4paper]{article} 
\input{head}

\usepackage{mathtools}
\usepackage{amsmath}
\usepackage[shortlabels]{enumitem}
\DeclarePairedDelimiter\abs{\lvert}{\rvert}%
\usepackage{cancel}

\begin{document}

%-------------------------------
%	TITLE SECTION
%-------------------------------

\fancyhead[C]{}
\hrule \medskip % Upper rule
\begin{minipage}{0.295\textwidth} 
\raggedright
\footnotesize
José Antonio Álvarez Ocete \hfill\\   
77553417Q \hfill\\
joseantonio.alvarezo@estudiante.uam.es
\end{minipage}
\begin{minipage}{0.4\textwidth} 
\centering 
\large 
Ejercicios\\ 
\normalsize 
Métodos Avanzados en Estadística\\ 
\end{minipage}
\begin{minipage}{0.295\textwidth} 
\raggedleft
\today\hfill\\
\end{minipage}
\medskip\hrule 
\bigskip
	
%-------------------------------
%	CONTENTS
%-------------------------------

\section*{Ejercicio 2.}

\textbf{Enunciado.} Sea $X_1, \ldots, X_n$ una muestra de $n$ observaciones \emph{iid} de una distribución $F$ con esperanza $\mu$ y varianza $\sigma^2$, y sea $X_1^*, \ldots, X_n^*$ una muestra de $n$ observaciones de la distribución empírica de la muestra original $F_n$. Calcula las siguientes cantidades: \\

\begin{enumerate}[a)]
	\item $\E_{F_n}(\bar X_n^*) := \E(\bar X_n^* | X_1, \ldots, X_n)$
	
	Esta esperanza nos supone un cálculo directo utilizando la linealidad de la esperanza:
	
	\[
		\E_{F_n}(\bar X_n^*) = \E_{F_n} \bigg( \frac{1}{n} \sum_{i=1}^n X_i^* \bigg) = \frac{1}{n} \sum_{i=1}^n \E_{F_n}(X_i^*) \stackrel{\text{iid}}{=} \frac{1}{n} \cdot n \ \E_{F_n}(X_1^*) = \bar x
	\]
	
	\item $\E_{F}(\bar X_n^*)$
	
	Por un proceso análogo al anterior podemos obtener que:
	
	\[
		\E_{F}(\bar X_n^*) = \E_{F} \bigg( \frac{1}{n} \sum_{i=1}^n X_i^* \bigg) = \frac{1}{n} \sum_{i=1}^n \E_{F}(X_i^*) \stackrel{\text{iid}}{=} \frac{1}{n} \cdot n \ \E_{F}(X_1^*) = \E_{F}(X_1^*)
	\]
	
	Sin embargo, $\E_{F}(X_1^*)$ no se puede calcular directamente pues depende de la muestra tomada $X_1, \ldots, X_n$. Podemos utilizar la \textbf{ley de la esperanza iterada}$\E[Y] = \E(\E(Y|X))$ y el valor calculado en el apartado anterior:
		
	\[
		\E_{F}(X_1^*) = \E_{F}(\E(X_1^* | X_1, \ldots, X_n)) = \E_{F}(\E_{F_n}(X_1^*)) = \E_{F}(\bar x) = \mu
	\]
	
	\item $\Var_{F_n}(\bar X_n^*) := Var(\bar X_n^* | X_1, \ldots, X_n)$
	
	Expresaremos esta varianza en función de la varianza muestra insesgada:
	
	\[
		s^2 = \frac{1}{n-1} \bigg( \sum_i x_i^2 - n \bar x^2 \bigg) = \frac{n}{n-1} \bigg( \frac{1}{n}\sum_i x_i^2 - \bar x^2 \bigg)
	\]
	
	Al ser insesgada, sabemos que $\E[s^2] = \sigma^2$. Esto nos será útil en el último apartado. Procedamos con el cálculo de la varianza:
		
	\[
		\begin{split}
			\Var_{F_n}(\bar X_n^*) & \stackrel{\text{iid}}{=} \frac{1}{n^2} \sum_{i=1}^n \Var_{F_n}(X_i^*) \\
			& \stackrel{\text{iid}}{=} \frac{1}{n^2} \cdot n \ \Var_{F_n}(X_1^*) \\
			& = \frac{1}{n} \bigg( \E_{F_n}( {X_1^*}^2 ) - \E_{F_n}( X_1^* )^2 \bigg) \\
			& = \frac{1}{n} \bigg( \frac{1}{n} \sum_{i=1}^n X_i^2 - \bar x^2 \bigg) \\
			& = \frac{n-1}{n^2} \underbrace{\frac{n}{n-1} \bigg( \frac{1}{n} \sum_{i=1}^n X_i^2 - \bar x^2 \bigg)}_{s^2} \\
			& = \frac{n-1}{n^2} s^2 \\
		\end{split}
	\]
	
	\item $Var_{F}(\bar X_n^*)$
	
	Para este último apartado haremos uso de la \textbf{ley de la varianza total}:
	
	\[
		\Var(Y) = \E(\Var(Y|X)) + \Var(\E(Y|X))
	\]
	
	Procedemos con el cálculo de la varianza:
	
	\[
		\begin{split}
			\Var_{F}(\bar X_n^*) & = \E_{F}( \underbrace{\Var_{F_n}(\bar X^*)}_{\frac{n-1}{n^2} s^2} ) + \Var_{F}( \underbrace{\E_{F_n}(\bar X^*)}_{\bar x} )\\
			& = \frac{n-1}{n^2} \E_{F}(s^2) + \Var_{F}( \bar x ) \\
			& \stackrel{\text{iid}}{=} \frac{n-1}{n^2} \sigma^2 + \frac{1}{n} \Var_{F}(X_1)  \\
			& = \frac{n-1}{n^2} \sigma^2 + \frac{1}{n} \sigma^2 \\
			& = \frac{2n-1}{n^2} \sigma^2 \\
		\end{split}
	\]
	
\end{enumerate}

\end{document}
