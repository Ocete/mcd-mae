\documentclass[a4paper]{article} 
\input{head}

\usepackage{mathtools}
\usepackage{amsmath}
\usepackage[shortlabels]{enumitem}
\DeclarePairedDelimiter\abs{\lvert}{\rvert}%
\usepackage{cancel}

\begin{document}

%-------------------------------
%	TITLE SECTION
%-------------------------------

\fancyhead[C]{}
\hrule \medskip % Upper rule
\begin{minipage}{0.295\textwidth} 
\raggedright
\footnotesize
José Antonio Álvarez Ocete \hfill\\   
77553417Q \hfill\\
joseantonio.alvarezo@estudiante.uam.es
\end{minipage}
\begin{minipage}{0.4\textwidth} 
\centering 
\large 
Entrega de problemas Boostrap\\ 
\normalsize 
Métodos Avanzados en Estadística\\ 
\end{minipage}
\begin{minipage}{0.295\textwidth} 
\raggedleft
\today\hfill\\
\end{minipage}
\medskip\hrule 
\bigskip
	
%-------------------------------
%	CONTENTS
%-------------------------------

\section*{Ejercicio 2.}

\textbf{Enunciado.} Sea $X_1, \ldots, X_n$ una muestra de $n$ observaciones \emph{iid} de una distribución $F$ con esperanza $\mu$ y varianza $\sigma^2$, y sea $X_1^*, \ldots, X_n^*$ una muestra de $n$ observaciones de la distribución empírica de la muestra original $F_n$. Calcula las siguientes cantidades: \\

\begin{enumerate}[a)]
	\item $\E_{F_n}(\bar X_n^*) := \E(\bar X_n^* | X_1, \ldots, X_n)$
	
	Esta esperanza nos supone un cálculo directo utilizando la linealidad de la esperanza:
	
	\[
		\E_{F_n}(\bar X_n^*) = \E_{F_n} \bigg( \frac{1}{n} \sum_{i=1}^n X_i^* \bigg) = \frac{1}{n} \sum_{i=1}^n \E_{F_n}(X_i^*) \stackrel{\text{iid}}{=} \frac{1}{n} \cdot n \ \E_{F_n}(X_1^*) = \bar x
	\]
	
	\item $\E_{F}(\bar X_n^*)$
	
	Por un proceso análogo al anterior podemos obtener que:
	
	\[
		\E_{F}(\bar X_n^*) = \E_{F} \bigg( \frac{1}{n} \sum_{i=1}^n X_i^* \bigg) = \frac{1}{n} \sum_{i=1}^n \E_{F}(X_i^*) \stackrel{\text{iid}}{=} \frac{1}{n} \cdot n \ \E_{F}(X_1^*) = \E_{F}(X_1^*)
	\]
	
	Sin embargo, $\E_{F}(X_1^*)$ no se puede calcular directamente pues depende de la muestra tomada $X_1, \ldots, X_n$. Podemos utilizar la \textbf{ley de la esperanza iterada}$\E[Y] = \E(\E(Y|X))$ y el valor calculado en el apartado anterior:
		
	\[
		\E_{F}(X_1^*) = \E_{F}(\E(X_1^* | X_1, \ldots, X_n)) = \E_{F}(\E_{F_n}(X_1^*)) = \E_{F}(\bar x) = \mu
	\]
	
	\item $\Var_{F_n}(\bar X_n^*) := Var(\bar X_n^* | X_1, \ldots, X_n)$
	
	Expresaremos esta varianza en función de la varianza muestra insesgada:
	
	\[
		s^2 = \frac{1}{n-1} \bigg( \sum_i x_i^2 - n \bar x^2 \bigg) = \frac{n}{n-1} \bigg( \frac{1}{n}\sum_i x_i^2 - \bar x^2 \bigg)
	\]
	
	Al ser insesgada, sabemos que $\E[s^2] = \sigma^2$. Esto nos será útil en el último apartado. Procedamos con el cálculo de la varianza:
		
	\[
		\begin{split}
			\Var_{F_n}(\bar X_n^*) & \stackrel{\text{iid}}{=} \frac{1}{n^2} \sum_{i=1}^n \Var_{F_n}(X_i^*) \\
			& \stackrel{\text{iid}}{=} \frac{1}{n^2} \cdot n \ \Var_{F_n}(X_1^*) \\
			& = \frac{1}{n} \bigg( \E_{F_n}( {X_1^*}^2 ) - \E_{F_n}( X_1^* )^2 \bigg) \\
			& = \frac{1}{n} \bigg( \frac{1}{n} \sum_{i=1}^n X_i^2 - \bar x^2 \bigg) \\
			& = \frac{n-1}{n^2} \underbrace{\frac{n}{n-1} \bigg( \frac{1}{n} \sum_{i=1}^n X_i^2 - \bar x^2 \bigg)}_{s^2} \\
			& = \frac{n-1}{n^2} s^2 \\
		\end{split}
	\]
	
	\item $Var_{F}(\bar X_n^*)$
	
	Para este último apartado haremos uso de la \textbf{ley de la varianza total}:
	
	\[
		\Var(Y) = \E(\Var(Y|X)) + \Var(\E(Y|X))
	\]
	
	Procedemos con el cálculo de la varianza:
	
	\[
		\begin{split}
			\Var_{F}(\bar X_n^*) & = \E_{F}( \underbrace{\Var_{F_n}(\bar X^*_n)}_{\frac{n-1}{n^2} s^2} ) + \Var_{F}( \underbrace{\E_{F_n}(\bar X^*_n)}_{\bar x} )\\
			& = \frac{n-1}{n^2} \E_{F}(s^2) + \Var_{F}( \bar x ) \\
			& \stackrel{\text{iid}}{=} \frac{n-1}{n^2} \sigma^2 + \frac{1}{n} \Var_{F}(X_1)  \\
			& = \frac{n-1}{n^2} \sigma^2 + \frac{1}{n} \sigma^2 \\
			& = \frac{2n-1}{n^2} \sigma^2 \\
		\end{split}
	\]
	
\end{enumerate}

\section*{Ejercicio 7.}

\textbf{Enunciado.} Sea $F$ una distribución con media $\mu$, varianza $\sigma^2$ y coeficiente de asimetía:
\[
	\gamma = \frac{\E[(X - \mu)^3]}{\sigma^3}
\]

Genera $R = 1000$ muestras de observaciones iid $X_1, \ldots, X_n$ con $X_i \equiv N(0, 1)$ para $n = 100$. Para cada una de ellas, calcula tres intervalos de confianza bootstrap de nivel 95\% para $\gamma$ usando el método híbrido, el método normal y el método percentil. Determina el porcentaje de intervalos que contienen al parámetro en cada caso. Repite el ejercicio con muestras procedentes de una distribución exponencial de parámetro $\lambda = 1$. \\

Resolveremos este ejercicio utilizando R. Se adjunta el cuaderno de Rmarkdown, en caso de que el resto del ejercicio sea más cómodo de visualizar en dicho formato puede utilizar Rstudio para ello. Importamos los paquetes necesarios:

\begin{Shaded}
	\begin{Highlighting}[]
\KeywordTok{library}\NormalTok{(tidyverse)}
\KeywordTok{library}\NormalTok{(gapminder)}
\KeywordTok{library}\NormalTok{(comprehenr)}
\KeywordTok{library}\NormalTok{(ggplot2)}
\KeywordTok{library}\NormalTok{(dplyr)}
\KeywordTok{theme_set}\NormalTok{(}\KeywordTok{theme_bw}\NormalTok{())}
	\end{Highlighting}
\end{Shaded}

Implementamos una función que calcula el coeficiente de asimetría
muestral de una muestra dada. Utilizaremos el estimador natural:
sustituir la esperanza por el promedio de los valores, la media por la
media muestral y la varianza por la varianza muestral, siendo
conscientes de que esta varianza muestra es \(s^2\) y no \(\sigma^2\).

\begin{Shaded}
	\begin{Highlighting}[]
\NormalTok{asymmetry_coefficient <-}\StringTok{ }\ControlFlowTok{function}\NormalTok{(muestra) \{}
\NormalTok{  mean <-}\StringTok{ }\KeywordTok{mean}\NormalTok{(muestra)}
\NormalTok{  sigma <-}\StringTok{ }\KeywordTok{sd}\NormalTok{(muestra)}
\KeywordTok{mean}\NormalTok{((muestra }\OperatorTok{-}\StringTok{ }\NormalTok{mean)}\OperatorTok{^}\DecValTok{3} \OperatorTok{/}\StringTok{ }\NormalTok{sigma)}
\NormalTok{\}}
	\end{Highlighting}
\end{Shaded}

Implementamos una función que pinta los intervalos alrededor del valor
real \(\theta\), sirviéndonos del código proporcionado en las
diapositivas.

\begin{Shaded}
	\begin{Highlighting}[]
\NormalTok{plot_intervals <-}\StringTok{ }\ControlFlowTok{function}\NormalTok{(intervals, hits, theta, method_name, distribution_name) \{}
\NormalTok{  m <-}\StringTok{ }\KeywordTok{nrow}\NormalTok{(intervals)}
\NormalTok{  df <-}\StringTok{ }\KeywordTok{data.frame}\NormalTok{(ic_min <-}\StringTok{ }\NormalTok{intervals[,}\DecValTok{1}\NormalTok{],}
\NormalTok{                   ic_max <-}\StringTok{ }\NormalTok{intervals[,}\DecValTok{2}\NormalTok{],}
\DataTypeTok{ind =} \DecValTok{1}\OperatorTok{:}\NormalTok{m,}
\DataTypeTok{hits =}\NormalTok{ hits)}

\NormalTok{  gg <-}\StringTok{ }\KeywordTok{ggplot}\NormalTok{(df) }\OperatorTok{+}
\StringTok{    }\KeywordTok{geom_linerange}\NormalTok{(}\KeywordTok{aes}\NormalTok{(}\DataTypeTok{xmin=}\NormalTok{ic_min, }\DataTypeTok{xmax=}\NormalTok{ic_max, }\DataTypeTok{y=}\NormalTok{ind, }\DataTypeTok{col=}\NormalTok{hits)) }\OperatorTok{+}
\StringTok{    }\KeywordTok{scale_color_hue}\NormalTok{(}\DataTypeTok{labels =} \KeywordTok{c}\NormalTok{(}\StringTok{"NO"}\NormalTok{, }\StringTok{"SÍ"}\NormalTok{)) }\OperatorTok{+}
\StringTok{    }\KeywordTok{geom_vline}\NormalTok{(}\KeywordTok{aes}\NormalTok{(}\DataTypeTok{xintercept =}\NormalTok{ theta), }\DataTypeTok{linetype =} \DecValTok{2}\NormalTok{) }\OperatorTok{+}
\StringTok{    }\KeywordTok{theme_bw}\NormalTok{() }\OperatorTok{+}
\StringTok{    }\KeywordTok{labs}\NormalTok{(}\DataTypeTok{y =} \StringTok{'Muestras'}\NormalTok{, }\DataTypeTok{x =} \StringTok{'Intervalos (nivel 0.95)'}\NormalTok{,}
\DataTypeTok{title =} \KeywordTok{paste}\NormalTok{(}\StringTok{'IC -'}\NormalTok{, method_name, }\StringTok{'para'}\NormalTok{, distribution_name, }\DataTypeTok{sep=}\StringTok{' '}\NormalTok{))}
\KeywordTok{print}\NormalTok{(gg)}
\NormalTok{\}}
	\end{Highlighting}
\end{Shaded}

Implementamos dos funciones auxiliares adicionales. La primera muestra
una distribución (normal o exponencial) \texttt{n} veces. La segunda
calcula el intervalo de confianza del coeficiente de asimetría
utilizando el método proporcionado por parámetro.

\begin{Shaded}
	\begin{Highlighting}[]
\NormalTok{sample_dist <-}\StringTok{ }\ControlFlowTok{function}\NormalTok{(distribution, n) \{}
\ControlFlowTok{if}\NormalTok{ (distribution }\OperatorTok{==}\StringTok{ 'normal'}\NormalTok{) \{}
\KeywordTok{rnorm}\NormalTok{(n, }\DecValTok{0}\NormalTok{, }\DecValTok{1}\NormalTok{)}
\NormalTok{  \} }\ControlFlowTok{else} \ControlFlowTok{if}\NormalTok{ (distribution }\OperatorTok{==}\StringTok{ 'exponential'}\NormalTok{) \{}
\KeywordTok{rexp}\NormalTok{(n, }\DecValTok{1}\NormalTok{)}
\NormalTok{  \} }\ControlFlowTok{else}\NormalTok{ \{}
\KeywordTok{print}\NormalTok{(}\StringTok{'Distribution not supported'}\NormalTok{)}
\NormalTok{  \}}
\NormalTok{\}}

\NormalTok{compute_interval <-}\StringTok{ }\ControlFlowTok{function}\NormalTok{(method, gammas_bootstrap, gamma_original, alfa) \{}
\ControlFlowTok{if}\NormalTok{ (method }\OperatorTok{==}\StringTok{ 'Método híbrido'}\NormalTok{) \{}
\CommentTok{# Metodo hibrido}
\NormalTok{    n =}\StringTok{ }\KeywordTok{length}\NormalTok{(gammas_bootstrap)}
\NormalTok{    T_bootstrap <-}\StringTok{ }\KeywordTok{sqrt}\NormalTok{(n) }\OperatorTok{*}\StringTok{ }\NormalTok{(gammas_bootstrap }\OperatorTok{-}\StringTok{ }\NormalTok{gamma_original)}
\NormalTok{    ic_min <-}\StringTok{ }\NormalTok{gamma_original }\OperatorTok{-}\StringTok{  }\KeywordTok{quantile}\NormalTok{(T_bootstrap, }\DecValTok{1}\OperatorTok{-}\NormalTok{alfa}\OperatorTok{/}\DecValTok{2}\NormalTok{)}\OperatorTok{/}\KeywordTok{sqrt}\NormalTok{(n)}
\NormalTok{    ic_max  <-}\StringTok{ }\NormalTok{gamma_original }\OperatorTok{-}\StringTok{  }\KeywordTok{quantile}\NormalTok{(T_bootstrap, alfa}\OperatorTok{/}\DecValTok{2}\NormalTok{)}\OperatorTok{/}\KeywordTok{sqrt}\NormalTok{(n)}
\KeywordTok{c}\NormalTok{(ic_min, ic_max)}
\NormalTok{  \} }\ControlFlowTok{else} \ControlFlowTok{if}\NormalTok{ (method }\OperatorTok{==}\StringTok{ 'Método normal'}\NormalTok{) \{}
\CommentTok{# Metodo normal}
\NormalTok{    et_boostrap <-}\StringTok{ }\KeywordTok{sd}\NormalTok{(gammas_bootstrap)}
\NormalTok{    ic_min <-}\StringTok{ }\NormalTok{gamma_original }\OperatorTok{+}\StringTok{ }\KeywordTok{qnorm}\NormalTok{(alfa}\OperatorTok{/}\DecValTok{2}\NormalTok{, }\DecValTok{0}\NormalTok{, }\DecValTok{1}\NormalTok{)}\OperatorTok{*}\NormalTok{et_boostrap}
\NormalTok{    ic_max  <-}\StringTok{ }\NormalTok{gamma_original }\OperatorTok{-}\StringTok{ }\KeywordTok{qnorm}\NormalTok{(alfa}\OperatorTok{/}\DecValTok{2}\NormalTok{, }\DecValTok{0}\NormalTok{, }\DecValTok{1}\NormalTok{)}\OperatorTok{*}\NormalTok{et_boostrap}
\KeywordTok{c}\NormalTok{(ic_min, ic_max)}
\NormalTok{  \} }\ControlFlowTok{else} \ControlFlowTok{if}\NormalTok{ (method }\OperatorTok{==}\StringTok{ 'Método percentil'}\NormalTok{) \{}
\CommentTok{# Metodo percentil}
\NormalTok{    ic_min <-}\StringTok{ }\KeywordTok{quantile}\NormalTok{(gammas_bootstrap, alfa}\OperatorTok{/}\DecValTok{2}\NormalTok{)}
\NormalTok{    ic_max  <-}\StringTok{ }\KeywordTok{quantile}\NormalTok{(gammas_bootstrap, }\DecValTok{1}\OperatorTok{-}\NormalTok{alfa}\OperatorTok{/}\DecValTok{2}\NormalTok{)}
\KeywordTok{c}\NormalTok{(ic_min, ic_max)}
\NormalTok{  \} }\ControlFlowTok{else}\NormalTok{ \{}
\KeywordTok{print}\NormalTok{(}\StringTok{'Method not supported'}\NormalTok{)}
\NormalTok{  \}}
\NormalTok{\}}
	\end{Highlighting}
\end{Shaded}

Utilizando las funciones anteriores implementamos una función final que,
dada una distibución y un método, calcula \texttt{n=100} muestras
originales, remuestrea \texttt{R=1000} veces y calcula el intervalor de
confianza para el coeficiente de asimetría. Repetimos este proceso
\texttt{m=1000} veces y dibujamos los distintos intervalos de confianza.
Adicionalmente se mostrará la precisión del método.

\begin{Shaded}
	\begin{Highlighting}[]
\NormalTok{exercise_}\DecValTok{2}\NormalTok{ <-}\StringTok{ }\ControlFlowTok{function}\NormalTok{ (distribution, method) \{}
\NormalTok{  R <-}\StringTok{ }\DecValTok{1000}
\NormalTok{  n <-}\StringTok{ }\DecValTok{100}
\NormalTok{  m <-}\StringTok{ }\DecValTok{1000}

\NormalTok{  alfa <-}\StringTok{ }\FloatTok{0.05}
\NormalTok{  theta =}\StringTok{ }\ControlFlowTok{if}\NormalTok{ (distribution }\OperatorTok{==}\StringTok{ 'normal'}\NormalTok{) }\DecValTok{0} \ControlFlowTok{else} \DecValTok{2}

\NormalTok{  intervalos <-}\StringTok{ }\OtherTok{NULL}
\NormalTok{  aciertos <-}\StringTok{ }\OtherTok{NULL}

\ControlFlowTok{for}\NormalTok{ (i }\ControlFlowTok{in} \DecValTok{1}\OperatorTok{:}\NormalTok{m) \{}
\CommentTok{# Usar la distribución para muestrear}
\NormalTok{    muestra_original <-}\StringTok{ }\KeywordTok{sample_dist}\NormalTok{(distribution, n)}
\NormalTok{    gamma_original <-}\StringTok{ }\KeywordTok{asymmetry_coefficient}\NormalTok{(muestra_original)}

\CommentTok{# Muestreo boostrap y computar los estimadores boostrap}
\NormalTok{    muestras_bootstrap <-}\StringTok{ }\KeywordTok{sample}\NormalTok{(muestra_original, n}\OperatorTok{*}\NormalTok{R, }\DataTypeTok{rep =} \OtherTok{TRUE}\NormalTok{)}
\NormalTok{    muestras_bootstrap <-}\StringTok{ }\KeywordTok{matrix}\NormalTok{(muestras_bootstrap, }\DataTypeTok{nrow =}\NormalTok{ n)}
\NormalTok{    gammas_bootstrap <-}\StringTok{ }\KeywordTok{apply}\NormalTok{(muestras_bootstrap, }\DecValTok{2}\NormalTok{, asymmetry_coefficient)}

\CommentTok{# Computar el intervalo y el acierto}
\NormalTok{    interv <-}\StringTok{ }\KeywordTok{compute_interval}\NormalTok{(method, gammas_bootstrap, gamma_original, alfa)}
\NormalTok{    intervalos <-}\StringTok{ }\KeywordTok{rbind}\NormalTok{(intervalos, interv)}
\NormalTok{    aciertos <-}\StringTok{ }\KeywordTok{c}\NormalTok{(aciertos, interv[}\DecValTok{1}\NormalTok{] }\OperatorTok{<}\StringTok{ }\NormalTok{theta }\OperatorTok{&}\StringTok{ }\NormalTok{interv[}\DecValTok{2}\NormalTok{] }\OperatorTok{>}\StringTok{ }\NormalTok{theta)}
\NormalTok{  \}}

\CommentTok{# Cálculo del accuracy}
\NormalTok{  acc =}\StringTok{ }\DecValTok{100} \OperatorTok{*}\StringTok{ }\KeywordTok{length}\NormalTok{(aciertos[aciertos }\OperatorTok\StringTok{ }\OtherTok{TRUE}\NormalTok{]) }\OperatorTok{/}\StringTok{ }\NormalTok{m}
	\KeywordTok{print}\NormalTok{(}\KeywordTok{paste}\NormalTok{(method, }\StringTok{' - accuracy: '}\NormalTok{, acc, }\StringTok{'%'}\NormalTok{, }\DataTypeTok{separator=}\StringTok{''}\NormalTok{))}
		
\CommentTok{# Gráfico}
\NormalTok{  distribution_name =}\StringTok{ }\ControlFlowTok{if}\NormalTok{ (distribution }\OperatorTok{==}\StringTok{ 'normal'}\NormalTok{) }\StringTok{'N(0,1))'} \ControlFlowTok{else} \StringTok{'exp(1)'}
\KeywordTok{plot_intervals}\NormalTok{(intervalos, aciertos, theta,}
\NormalTok{                 method, distribution_name)}
\NormalTok{\}}
			\end{Highlighting}
		\end{Shaded}
		
		Utilizamos la función anterior para mostrar los distintos intervalos de
		confianza para la normal, así como su precisión.
		
		\begin{Shaded}
			\begin{Highlighting}[]
\KeywordTok{set.seed}\NormalTok{(}\DecValTok{123}\NormalTok{)}

\NormalTok{methods <-}\StringTok{ }\KeywordTok{c}\NormalTok{(}\StringTok{'Método híbrido'}\NormalTok{, }\StringTok{'Método normal'}\NormalTok{, }\StringTok{'Método percentil'}\NormalTok{)}
\ControlFlowTok{for}\NormalTok{ (method }\ControlFlowTok{in}\NormalTok{ methods) \{}
\KeywordTok{exercise_2}\NormalTok{(}\StringTok{'normal'}\NormalTok{, method)}
\NormalTok{\}}
			\end{Highlighting}
		\end{Shaded}
		
		\begin{verbatim}
			## [1] "Método híbrido  - accuracy:  95.2 % "
		\end{verbatim}
		
		\includegraphics{figures/norm1.png}
		
		\begin{verbatim}
			## [1] "Método normal  - accuracy:  94.8 % "
		\end{verbatim}
		
		\includegraphics{figures/norm2.png}
		
		\begin{verbatim}
			## [1] "Método percentil  - accuracy:  94.6 % "
		\end{verbatim}
		
		\includegraphics{figures/norm3.png}
		
		Obtenemos así precisiones cercanas al 95\%. Este resultado era de
		esperar pues tomamos \texttt{alfa=0.05}. Repetimos el experimento para
		la distribución exponencial con parámetro \(\lambda=1\).
		
		\begin{Shaded}
			\begin{Highlighting}[]
\NormalTok{methods <-}\StringTok{ }\KeywordTok{c}\NormalTok{(}\StringTok{'Método híbrido'}\NormalTok{, }\StringTok{'Método normal'}\NormalTok{, }\StringTok{'Método percentil'}\NormalTok{)}
\ControlFlowTok{for}\NormalTok{ (method }\ControlFlowTok{in}\NormalTok{ methods) \{}
\KeywordTok{exercise_2}\NormalTok{(}\StringTok{'exponential'}\NormalTok{, method)}
\NormalTok{\}}
			\end{Highlighting}
		\end{Shaded}
		
		\begin{verbatim}
			## [1] "Método híbrido  - accuracy:  65.3 % "
		\end{verbatim}
		
		\includegraphics{figures/exp1.png}
		
		\begin{verbatim}
			## [1] "Método normal  - accuracy:  68 % "
		\end{verbatim}
		
		\includegraphics{figures/exp2.png}
		
		\begin{verbatim}
			## [1] "Método percentil  - accuracy:  63.6 % "
		\end{verbatim}
		
		\includegraphics{figures/exp3.png}
		
		En este caso, el coeficiente de asimetría real es 2. Sin embargo, vemos
		como obtenemos una precisión mucho menor: entorno al 65\% en los
		diferentes métodos. Esto puede deberse a que el coeficiente de asimetría
		no se estime correctamente utilizando boostrap.

\end{document}
